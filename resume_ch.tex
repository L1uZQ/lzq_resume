%%%%%%%%%%%%%%%%%%%%%%%%%%%%%%%%%%%%%%%%%
% Medium Length Professional CV
% LaTeX Template
% Version 2.0 (8/5/13)
%
% This template has been downloaded from:
% http://www.LaTeXTemplates.com
%
% Original author:
% Trey Hunner (http://www.treyhunner.com/)
%
% Important note:
% This template requires the resume.cls file to be in the same directory as the
% .tex file. The resume.cls file provides the resume style used for structuring the
% document.
%
%%%%%%%%%%%%%%%%%%%%%%%%%%%%%%%%%%%%%%%%%

%----------------------------------------------------------------------------------------
%	PACKAGES AND OTHER DOCUMENT CONFIGURATIONS
%----------------------------------------------------------------------------------------

% !TeX program = xelatex
\documentclass{resume} % Use the custom resume.cls style
\usepackage[left=0.75in,top=0.6in,right=0.75in,bottom=0.6in]{geometry} % Document margins
\usepackage{xeCJK}
\usepackage{color}
\usepackage[citecolor=green]{hyperref}
%导入图片


\name{刘 振强} % Your name
\address{男 \quad  年龄:20 \quad 求职意向:C++开发/前端开发}
\address{成都市高新区(西区)西源大道2006号电子科技大学清水河校区 \qquad   邮编:611731}  % Your address
\address{+(86) 13643738648 \\ uestclzq@qq.com} % Your phone number and email


\begin{document}

%----------------------------------------------------------------------------------------
%	EDUCATION SECTION
%----------------------------------------------------------------------------------------

\begin{rSection}{教育经历}

{\textbf{电子科技大学 UESTC}} \hfill {\em 2020.9 - 至今} \\
2020级本科生,大三在读,计算机科学与技术(计算机+金融双学位培养)

\end{rSection}


%----------------------------------------------------------------------------------------
%	SKILLS SECTION
%----------------------------------------------------------------------------------------

\begin{rSection}{技能}
\begin{rSubsection}
{}{}{}{}
\item[-] 编程语言: 熟悉 C/C++,熟悉STl模板库的使用,了解C++并发与锁相关知识\\
 了解Go,了解shell、node.js,了解前端基础和Vue框架
\item[-] 计算机基础: 熟悉计算机组成和结构,熟悉常见的数据结构和算法,熟悉操作系统基础,了解多线程与网络编程。
\item[-] 其它: 熟悉Linux常用操作和命令,了解docker的使用,能熟练使用Git进行版本控制,熟悉 MySQL数据库与openGauss云数据库。
\end{rSubsection}
\end{rSection}


%----------------------------------------------------------------------------------------
%	PROJECTS / RESEARCH EXPERIENCE SECTION
%----------------------------------------------------------------------------------------

\begin{rSection}{项目 / 研究经历}
\begin{rSubsection}{1.腾讯游戏客户端安全菁英班(电子科大30人班)}{\em 2022.10 – 2023.3}
{\textbf{项目介绍:}腾讯老师线上授课(内容为游戏客户端安全与逆向分析基础)+完成课后项目,课后项目包括对游戏FlappyBird进行注入和修改、对APP显示内容进行HOOK、破解一个ELF文件,以及为一个PFS游戏实现游戏透视与自瞄。} \
{\textbf{项目收获:}掌握了windows和linux平台进行注入的方式,了解一些逆向分析工具如CE、IDA、ReClass、的使用,掌握了使用IDA对远程进程进行动态调试,以及通过实战了解了FPS游戏透视和自瞄的简单实现。} \\
{\textbf{项目成果:}} 
\item[]
\begin{itemize}
\setlength\itemsep{-0.5em}
\item[] 
%\item[-] 通过动态调试破解一个加密ELF文件:  \  \rm \url{https://blog.csdn.net/astruggle___/article/details/129377398?spm=1001.2014.3001.5501}
%\item[-] 为游戏AssaultCube实现透视和自瞄:  \  Github: \rm \url{https://github.com/L1uZQ/Assaultcube_Cheat_onlyStudy}
\end{itemize}
\end{rSubsection}

\begin{rSubsection}{1.腾讯游戏客户端安全菁英班(电子科大30人班)}{\em 2022.10 – 2023.3}
{\textbf{项目介绍:}腾讯老师线上授课(内容为游戏客户端安全与逆向分析基础)+完成课后项目,课后项目包括对游戏FlappyBird进行注入和修改、对APP显示内容进行HOOK、破解一个ELF文件,以及为一个PFS游戏实现游戏透视与自瞄。} \
{\textbf{项目收获:}掌握了windows和linux平台进行注入的方式,了解一些逆向分析工具如CE、IDA、ReClass、的使用,掌握了使用IDA对远程进程进行动态调试,以及通过实战了解了FPS游戏透视和自瞄的简单实现。} \\
{\textbf{项目成果:}} 
\item[]
\begin{itemize}
\setlength\itemsep{-0.5em}
\item[] 
%\item[-] 通过动态调试破解一个加密ELF文件:  \  \rm \url{https://blog.csdn.net/astruggle___/article/details/129377398?spm=1001.2014.3001.5501}
%\item[-] 为游戏AssaultCube实现透视和自瞄:  \  Github: \rm \url{https://github.com/L1uZQ/Assaultcube_Cheat_onlyStudy}
\end{itemize}
\end{rSubsection}




\begin{rSubsection}{2.小米暑期快应用训练营} {\em 2022.6 - 2022.8}
%{\textbf{项目介绍:}先由小米老师讲授前端三件套加快应用开发基础,之后个人独立开发一个具有震动提醒和kaldi语音识别的TodoList待办事项快应用。} \\
{\noindent \textbf{项目介绍:}腾讯老师线上授课(内容为游戏客户端安全与逆向分析基础)+完成课后项目,课后项目包括对游戏FlappyBird进行注入和修改、对APP显示内容进行HOOK、破解一个ELF文件,以及为一个PFS游戏实现游戏透视与自瞄。} \
{项目收获:掌握了windows和linux平台进行注入的方式,了解一些逆向分析工具如CE、IDA、ReClass、的使用,掌握了使用IDA对远程进程进行动态调试,以及通过实战了解了FPS游戏透视和自瞄的简单实现。} 
{}
%{\textbf{项目收获:}了解了前端基础,掌握了安卓平台一些接口的调用方式,通过实战的方式了解了js定时器的应用,了解了一些开发规范和注意事项。} \
%{\textbf{项目成果:} 快简记- Github: \rm \url{https://github.com/L1uZQ/todolist-project}} 
\item[]
\begin{itemize}
\setlength\itemsep{-0.5em}
\item[]
\item[-] 支持修改待办事项
%\item[-] 使用 kaldi接口 进行语音输入待办事项
%\item[-] 到 deadline 了进行震动提醒
%\item[-] 状态可以自动更新
\end{itemize}
\end{rSubsection}


\begin{rSubsection}{3.C++ TinyWebServer} {\em 2023.1 - 2023.3}
{\\}
{用C++实现了一个基于(线程池+epoll复用+主从状态机)的简易HTTP服务器,总共1000+行。}
\item[]
\begin{itemize}
\setlength\itemsep{-0.5em}
\item[-] 压测:腾讯云2核2G2M轻量云服务器,webbench测得10s内800次并发
\end{itemize}
{Github: \rm \url{https://github.com/L1uZQ/lzq_TinyWebServer}}
\end{rSubsection}


\end{rSection}



%----------------------------------------------------------------------------------------
%	HONORS / AWARDS SECTION
%----------------------------------------------------------------------------------------

\begin{rSection}{荣誉 / 奖项}
\begin{tabular}{ @{} >{\bfseries}l @{\hspace{6ex}} l }
全国大学生数学竞赛(四川赛区)二等奖 & {\em 2021.12} \\
国家励志奖学金 & {\em 2022} \\
校优秀学生奖学金一等 & {\em 2022} \\
第十一届中国卫星导航年会表现优秀志愿者 & {\em 2020.12} \\
\end{tabular}
\end{rSection}

%----------------------------------------------------------------------------------------
%	WORKING EXPERIENCE SECTION
%----------------------------------------------------------------------------------------



%----------------------------------------------------------------------------------------
%	OTHERS SECTION
%----------------------------------------------------------------------------------------

\begin{rSection}{其他}
\begin{rSubsection}
{}{}{}{}
\item[-] GitHub: https://github.com/L1uZQ
\item[-] 数学水平: 数学相关课程均满绩,平均分:89 \quad  英语水平: CET-6:511  CET-4:571.
\item[-]专业课GPA: 3.85/4.00 \quad 目前全程成绩单:\rm \url{https://maifile.cn/est/a2606781672791/pdf}
\end{rSubsection}
\end{rSection}



\end{document}
