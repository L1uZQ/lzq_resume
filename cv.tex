%%%%%%%%%%%%%%%%%%%%%%%%%%%%%%%%%%%%%%%%%
% Medium Length Professional CV
% LaTeX Template
% Version 2.0 (8/5/13)
%
% This template has been downloaded from:
% http://www.LaTeXTemplates.com
%
% Original author:
% Trey Hunner (http://www.treyhunner.com/)
%
% Important note:
% This template requires the resume.cls file to be in the same directory as the
% .tex file. The resume.cls file provides the resume style used for structuring the
% document.
%
%%%%%%%%%%%%%%%%%%%%%%%%%%%%%%%%%%%%%%%%%

%----------------------------------------------------------------------------------------
%	PACKAGES AND OTHER DOCUMENT CONFIGURATIONS
%----------------------------------------------------------------------------------------

\documentclass{resume} % Use the custom resume.cls style
%\usepackage[left=0.75in,top=0.6in,right=0.75in,bottom=0.6in]{geometry} % Document margins
\usepackage[left=0.65in,top=0.6in,right=0.65in,bottom=0.6in]{geometry} % Document margins
\usepackage{color}
\usepackage[citecolor=green
            ]{hyperref}

\name{Xxxxxx Xxx} % Your name
\address{+(86) 132xxxxx975 \\ xxxxxx@gmail.com} % Your phone number and email
\address{Peking University, Haidian District, Beijing 100871, China}  % Your address
\begin{document}

%----------------------------------------------------------------------------------------
%	EDUCATION SECTION
%----------------------------------------------------------------------------------------

\begin{rSection}{Education}
{\textbf{Peking University}} \hfill {\em September 2016 - present} \\ 
M.S. in Peking University, Institute of Network, GPA:xx(100)

{\textbf{Peking University}} \hfill {\em September 2012 - June 2016} \\
B.S. in Peking University, Department of Computer Science
\begin{rSubsection}{}{}{}{}
\item Total GPA: xx/4.00 \ \ Major GPA: xx/4.00 
\end{rSubsection}

{\textbf{Edinburgh University}} \hfill {\em June 2015 - September 2015} \\ 
Visiting Scholar in School of Informatics

\end{rSection}

%----------------------------------------------------------------------------------------
%	HONORS / AWARDS SECTION
%----------------------------------------------------------------------------------------

\begin{rSection}{Honors / Awards}
\begin{tabular}{ @{} >{\bfseries}l @{\hspace{6ex}} l }
NOI – First Prize & {\em July 2011} \\
ACM/ACPC Tianjin regional competition – Second Prize & {\em September 2012} \\
Second Prize in ACM Programming Contest (on campus) & {\em May 2013} \\
Third Prize in ACM Programming Contest (on campus) & {\em May 2014} \\
Winner in Peking University AI Against Competition & {\em June 2015} \\
Runner-up in SIGMOD2017 Programming Contest & {\em March 2017} \\
54-Youth Scholarship & {\em 2014} \\
\end{tabular}
\end{rSection}

%----------------------------------------------------------------------------------------
%	WORKING EXPERIENCE SECTION
%----------------------------------------------------------------------------------------

\begin{rSection}{Working Experience}

\begin{rSubsection}{Microsoft STC Asia}{\em September 2017 - February 2018}
{Internship, Department of Xiaoice \\}
{Responsible for exploring new features in QA process and some other system coding tasks in C\#.}
\item[]
\begin{itemize}
\setlength\itemsep{-0.5em}
\item[-] Implementation of CNN network in C\# which 2 times faster than original version under Intel-MKL Library and 40 times faster than original version under memory cache.
\item[-] Finding a new approach to get the matched sentence in an article with its comment.
\item[-] Implementing some pipelines deal with databases and ElasticSearch servers.
\end{itemize}
\end{rSubsection}

%------------------------------------------------

\begin{rSubsection}{Bytedance Inc. Beijing}{\em June 2017 - August 2017}
{Internship, Department of AI-Lab \\}
{Responsible for maintaining the server side of video deduplication under Thrift-Framework in C++.}
\item[]
\begin{itemize}
\setlength\itemsep{-0.5em}
\item[-] Add other features in video duduplication process, such as deep-learning feature, orb feature and k-means based feature.
\item[-] Implementation of a web page tool to label the videos manually.
\item[-] Tasks of data processing.
\end{itemize}
\end{rSubsection}

%------------------------------------------------

\begin{rSubsection}{Megvii Inc. Beijing}{\em March 2016 - September 2016}
{Internship, Team of Text Recognization \\}
{Responsible for maintaining the pipeline of using the model to recognize all characters of a picture in C++.}
\item[]
\begin{itemize}
\setlength\itemsep{-0.5em}
\item[-] Implementation of minAreaRect with C++.
\item[-] Implementation of text in lines with minimum-cost flow.
\item[-] Responsible for adjusting pipeline according to different models, and displaying demo on website.
\end{itemize}
\end{rSubsection}

\end{rSection}

%----------------------------------------------------------------------------------------
%	PROJECTS / RESEARCH EXPERIENCE SECTION
%----------------------------------------------------------------------------------------

\begin{rSection}{Projects / Research Experience}

\begin{rSubsection}{Automatic Hyperparameter Tuning} {\em December 2017 - January 2018}
{Implementation of a robust Bayesian Optimization framework written in Java. About 2000 lines.\\}
{Github: \rm \url{https://github.com/xxxxxxx}}
\item[]
\begin{itemize}
\setlength\itemsep{-0.5em}
\item[-] Model: Gaussian Process.
\item[-] Maximizer: Random Sampling.
\item[-] Acquisition Function: EI, LCB, GP-UCB
\end{itemize}
\end{rSubsection}

\begin{rSubsection}{SIGMOD 2017 Programming Contest}{\em January 2017 – March 2017}{2-people Semi-final Rank(1/40). Final Rank(2/40) \\}
{Given the n-grams, documents and many remove and add operations, the task is finding all n-grams in each
document. The contest is ranked by time.}
\item[]
\begin{itemize}
\setlength\itemsep{-0.5em}
\item[-] Using Hash Table and Trie to make index.
\item[-] Achieving complete parallel to insert, delete and query operations.
\item[-] Codes: about 3000 lines in C++.
\end{itemize}
\end{rSubsection}

\begin{rSubsection}{Alibaba Big Data Contest} {\em October 2016 - December 2016}
{Prediction of offline o2o coupon tickets using Ranking in preliminary(1/1500). Ranking in semi-final(19/1500) \\}
{Three-people team, responsible for the choice of features and models, data cleaning, etc.}
\item[]
\begin{itemize}
\setlength\itemsep{-0.5em}
\item[-] Learning Ali Shujia in semi-final, writing MR process to extract features with java.
\item[-] The model of LR, Random Forest, GBDT and fusion of various models are tried.
\item[-] Temporal features were added and AUC raised by 1\%.
\end{itemize}
\end{rSubsection}

\begin{rSubsection}{3D Online Mobile Game }{\em July 2016 – August 2016}{Independent development \\}
{A MOBA game on Android, each game supports the maximum of 4 people battling together. The game supports
players combating according to capability level, provides user login system, and stores personal information in
the server.}
\item[]
\begin{itemize}
\setlength\itemsep{-0.5em}
\item[-] The game logic part uses the Unity3D engine with C\#.
\item[-] Server side is rented Ali Cloud Server, using Django framework written, Apache deployment.
\item[-] Codes: about 7000 lines.
\end{itemize}
\end{rSubsection}

\begin{rSubsection}{Java Comment Matchment}{\em July 2015 – August 2015}{Exchanging project to Edinburgh University 2 months \\}
{Using deep learning to match java code blocks and comment blocks with professor Charles Sutton. The ultimate
goal is to achieve code completion automatically}
\item[]
\begin{itemize}
\setlength\itemsep{-0.5em}
\item[-] The dataset is the java project on Github.
\item[-] Writing java tokenizer to extract java code token, separating code blocks and comment blocks.
\item[-] Making training data: experimenting various rules to match the code blocks with the comment blocks.
\end{itemize}
\end{rSubsection}

\end{rSection}

%----------------------------------------------------------------------------------------
%	SKILLS SECTION
%----------------------------------------------------------------------------------------

\begin{rSection}{Skills}
\begin{rSubsection}
{}{}{}{}
\item[-] Programming language: proficient in C/C++, C\#, java and python, having knowledge of shell, matlab and some
functional programming languages like scala and scheme
\item[-] Big data: Familiar with Hadoop and Spark.
\item[-] Machine learning:proficient in numpy, matplotlib, sklearn and xgboost, having knowledge of pandas and
tensorflow
\end{rSubsection}
\end{rSection}

%----------------------------------------------------------------------------------------
%	OTHERS SECTION
%----------------------------------------------------------------------------------------

\begin{rSection}{Others}
\begin{rSubsection}
{}{}{}{}
\item[-] Maths Level: A and A+ in Mathematic Analysis, Advanced Algebra, Discrete Mathematics.
\item[-] English Level: CET-6.
\item[-] GitHub: https://github.com/xxxxxxx
\end{rSubsection}
\end{rSection}

\end{document}
